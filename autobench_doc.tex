\documentclass[10pt, a4paper]{report}
\usepackage[backend=bibtex,style=numeric]{biblatex}
\usepackage[justification=centering]{caption}
\usepackage[pdfpagelabels,bookmarks,hyperindex,hyperfigures, hidelinks]{hyperref}
\usepackage[toc, page]{appendix}
\usepackage[utf8]{inputenc}
\usepackage[version=4]{mhchem}
\usepackage{alltt}
\usepackage{amsmath}
\usepackage{booktabs}
\usepackage{circuitikz}
\usepackage{comment}
\usepackage{csquotes}
\usepackage{epstopdf}
\usepackage{fancyhdr}
\usepackage{gensymb}
\usepackage{hyphsubst}
\usepackage{lastpage}
\usepackage{multirow}
\usepackage{parskip}
\usepackage{pdfpages}
\usepackage{placeins}
\usepackage{subcaption}
\usepackage{tabularx}
\usepackage{textcomp}
\usepackage{tikz}
\usepackage{titlesec}
\usepackage{upgreek}
\usepackage{wrapfig}
\usepackage{xfrac}
\usepackage{pdfpages}
\usepackage{multicol}

\usetikzlibrary{arrows, arrows.meta, decorations.markings, positioning}
\addbibresource{autobench_doc.bib}

% if definition for section compilling control
\newif\ifdebug
%\debugtrue
\debugfalse

\setlength{\parskip}{0.6em}

% Set the indentation length 
\setlength{\parindent}{0em}


% Glossary entries
\usepackage[acronym, nomain]{glossaries}
\makenoidxglossaries

\newacronym{CPU}{CPU}{Central Processing Unit}
\newacronym{DCLS}{DCLS}{Dual core Lockstep}
\newacronym{DD}{DD}{Displacement Damage}
\newacronym{DLD}{DLD}{Deep Level detector}
\newacronym{ECC}{ECC}{Error Correction Codes}
\newacronym{FI}{FI}{Functional Interrupts}
\newacronym{FTM}{FTM}{Fault Tolerance Monitor}
\newacronym{MCU}{MCU}{Microcontroller Unit}
\newacronym{NMR}{NMR}{N-Modular Redundancy}
\newacronym{PC}{PC}{Program Counter}
\newacronym{QOS}{QOS}{Quality of Service}
\newacronym{SDC}{SDC}{Silent Data Corruption}
\newacronym{SEEs}{SEEs}{Single-Event Effects}
\newacronym{SEFIs}{SEFIs}{Single-Event Functional Interrupts}
\newacronym{SET}{SET}{Single-Event Transient}
\newacronym{SEU}{SEU}{Single-Even Upset}
\newacronym{TID}{TID}{Total Ionizing Dose}
\newacronym{TMR}{TMR}{Triple Modular Redundancy}
\newacronym{DWC}{DWC}{Duplication with Comparisson}
\newacronym{SIHFT}{SIHFT}{Software-implemented hardware fault tolerance}
\newacronym{ABFT}{ABFT}{Algorithm-Based Fault Tolerance}
\newacronym{HETA}{HETA}{Hybrid Error-Detection Technique Using Assertions}
\newacronym{SSETA}{S-SETA}{Selective Software-Only Error-Detection Technique
Using Assertions}
\newacronym{SWIFT}{SWIFT}{Software Implemented Fault Tolerance}
\newacronym{SWIFTR}{SWIFT-R}{Software Implemented Fault Tolerance with Recovery}
\newacronym{SSWIFTR}{S-SWIFT-R}{Selective Software Implemented Fault Tolerance 
with Recovery}
\newacronym{IP}{IP}{Intellectual Property}
\newacronym{ISA}{ISA}{Instruction Set Architecture}


\tikzstyle{vecArrow} = [thick, decoration={markings,mark=at position
	1 with {\arrow[semithick]{open triangle 60}}},
double distance=1.4pt, shorten >= 5.5pt,
preaction = {decorate},
postaction = {draw,line width=1.4pt, white,shorten >= 4.5pt}]
\tikzstyle{innerWhite} = [semithick, white,line width=1.4pt, shorten >= 4.5pt]
\graphicspath{{assets/}} 
\newcommand\reaction[1]{\begin{equation}\ce{#1}\end{equation}} 

\titleclass{\subsubsubsection}{straight}[\subsection]

\newcounter{subsubsubsection}[subsubsection]
\renewcommand\thesubsubsubsection{\thesubsubsection.\arabic{subsubsubsection}}
\renewcommand\theparagraph{\thesubsubsubsection.\arabic{paragraph}} % optional; useful if paragraphs are to be numbered

\titleformat{\subsubsubsection}
  {\normalfont\normalsize\bfseries}{\thesubsubsubsection}{1em}{}
\titlespacing*{\subsubsubsection}
{0pt}{3.25ex plus 1ex minus .2ex}{1.5ex plus .2ex}

\makeatletter
\renewcommand\paragraph{\@startsection{paragraph}{5}{\z@}%
  {3.25ex \@plus1ex \@minus.2ex}%
  {-1em}%
  {\normalfont\normalsize\bfseries}}
\renewcommand\subparagraph{\@startsection{subparagraph}{6}{\parindent}%
  {3.25ex \@plus1ex \@minus .2ex}%
  {-1em}%
  {\normalfont\normalsize\bfseries}}
\def\toclevel@subsubsubsection{4}
\def\toclevel@paragraph{5}
\def\toclevel@paragraph{6}
\def\l@subsubsubsection{\@dottedtocline{4}{7em}{4em}}
\def\l@paragraph{\@dottedtocline{5}{10em}{5em}}
\def\l@subparagraph{\@dottedtocline{6}{14em}{6em}}
\makeatother

\setcounter{secnumdepth}{4}
\setcounter{tocdepth}{4}

\setcounter{MaxMatrixCols}{20}

\setlength{\headheight}{52pt}
\pagestyle{fancy}
\fancyheadoffset{0.5cm}
\fancyhead{}
\fancyhead[L]{\includegraphics[scale=0.3]{ida.png}}
\fancyhead[C]{Technische Universität Braunschweig\\
Institut für Datentechnik und Kommunikationsnetzte}
\fancyhead[R]{\includegraphics[scale=0.2]{tub.png}}

\renewcommand{\figurename}{Fig.}

\renewcommand\footrule{\begin{minipage}{0.95\textwidth}
    \hrule width \hsize   
\end{minipage}\par}

\setlength{\footskip}{30pt}
\fancyfoot[L]{\textit{M. Moya}}
\fancyfoot[C]{}
\fancyfoot[R]{\textit{Page \thepage{} de \pageref{LastPage}}}

\DeclareGraphicsExtensions{.bmp, .png, .jpg}

\topmargin = -1.5cm
\leftmargin = -1cm
\oddsidemargin = 0cm
\textheight = 24cm
\textwidth = 17cm

% renew itemize symbol level II
\renewcommand{\labelitemii}{$>$}

\begin{document}

\ifdebug
DebugOn
\newpage
\else
\begin{titlepage}
    \begin{center}
	\begin{minipage}{.45\textwidth}
	    \flushleft
	    \includegraphics[scale=0.5]{ida.png}
	\end{minipage}%
	\begin{minipage}{.45\textwidth}
	    \flushright
	    \includegraphics[scale=0.5]{tub.png}
	\end{minipage}

	\vspace{15mm}

	\large{ \textbf{Technische Universität Braunschweig}} \\[5mm]
	\textbf{Institut für Datentechnik und Kommunikationsnetze} \\[50mm]
	\Large{\textbf{Error Handling within a Dual Core Lockstep RISC-V Processor
    Architecture}} \\[15mm]

    \end{center}

    {\large\textbf{Author(s):}}
    \begin{itemize}
        \item [] Martin Moya - \texttt{m.moya@tu-braunschweig.de}
    \end{itemize}
    \vspace{10pt}
    {\large\textbf{Supervisor(s):}}

    \begin{itemize}
        \item [] Alexander Dörflinger - \texttt{adoerflinger@ida.ing.tu-bs.de}
    \end{itemize}
    \vspace{10pt}
    {\large\textbf{Examiner(s):}}
    \begin{itemize}
        \item [] Prof. Dr.-Ing. Rolf ernst - \texttt{ernst@ida.ing.tu-bs.de}
        \item [] Prof. Dr.-Ing. Harald Michalik - \texttt{michalik@ida.ing.tu-bs.de}
    \end{itemize}
    \vspace{10pt}
\end{titlepage}

\newpage

\begin{abstract}
    \thispagestyle{fancy}
    Microcontrollers (\acrshort{MCU}) are widely used in critical applications 
    due to low-energy consumption and high-performance computing power. Despite 
    these advantages, \acrshort{MCU}s are sensitive to radiation like any other
    electronic device, leading to transient and interminent faults causing
    cathastrophic situations.

    Critical applications have to function in a proper manner and deliver high
    level of \acrlong{QOS} (\acrshort{QOS}), on the other hand, these kind of
    applications have also strict time and cost constrains, which means that
    they do not only have to meet high \acrshort{QOS} standards, they also have
    to satisfy with a handfull of constraints. This work analyzes and proposes 
    the development of a software solution for error handling within a Dual Core 
    Lockstep (\acrshort{DCLS}) RISC-V Processor Architecture. The solution 
    provides a framework to implement different error handling techniques given 
    specific scenarios in order to satisfy both requirements.
\end{abstract}

\newpage
% Print table of contents
\begin{tableofcontents}
    \thispagestyle{fancy}
\end{tableofcontents}

\newpage

\chapter{Introduction}

\thispagestyle{fancy}
A system is considered \emph{Safety-critical} when a failure in such could
result in loss of life, significant property damage, or damage to the
environment. Aircrafts, cars, weapons systems, medical devices and nuclear 
plants are considered traiditional examples of safety-critical systems. Most of 
these applications, if not all, rely on embedded systems that are expected to be 
fault tolerant. A system fault tolerant, according to the authors in 
\ref{nasa_fault_tolerance}, is a system that is able to continue operating 
without interruption when one or more of its components fail, they aim 
to provide the ability to deliver a service that can be trusted, while fault 
removal and fault forecasting aim to reach confidence in that ability by 
justifying that the functional and the dependability and security specifications 
are adequate and that the systems is likely to meet them.

The objective of creating a fault-tolerant system is to prevent disruptions
arising from a single point of failure, ensuring the high availability and
business continuity of mission-critical applications or systems. As of now 
faults in distributed embedded systems can be permanent, intermittent or 
transient. Permanent faults cause long-term malfunctioning of components, while 
transient and intermittent faults appear for a short time. The effects of these 
faults, independently of their nature, can be devastating. They may corrupt data
or lead to logic miscalculations, which can result in a fatal failure or 
dramatic \acrshort{QOS} deterioration if not handled propperly. 

Transient and intermittent faults can be addressed in \emph{hardware} with
hardening techniques or in \emph{software}, using error handling techniques. 
Safety-critical applications have to be implemented such that they satisfy 
strict timing requirements and tolerate faults without exceeding a given amount 
of resources. Moreover, not only timeliness, reliability and cost-related 
requirements have to be considered but also other issues such as debugability 
and testability have to be taken into account. 



In this chapter, we motivate the importance of software error handling
techniques for fault tolerant systems during the design of these safety-critical
applications based on embedded systems. We introduce the starting point of the
proyect as well as the main contribution of our work. Finally, we present a
short overview of the document's structure.

\section{Motivation}

In this section we discuss the main sources of motivation for analyzing and
implementing different software error handling techniques for 

\newpage

% Print table of acronyms
\addcontentsline{toc}{section}{Tabla de Abreviaturas}
\glsaddall
\printnoidxglossary[type=\acronymtype,title={Abreviaturas}]

\end{document}
